近年普及の進む掃除ロボットを始めとする室内移動系のロボットは、車輪径・車高を大きくすることが難しいケースも少なくない。
しかし、一般的な車輪ロボットでは、車輪半径の半分以上、車高以上の段差を登攀することは困難である。
そのため、現行の室内移動ロボットの運用は、大きな段差のないフロアを移動することが主である。

% 画像
begin{figure}[htbp]
  centering
  includegraphics[width=50mm]{imageroomba.png}
  caption{段差に乗り上げるルンバ}
  label{figexample}
  end{figure}

したがって、室内移動型のロボットを複数階建ての一軒家などで運用する場合、
複数のフロアで稼働させるためには人間がロボットを移動させるか、別フロアの作業は人間が行う必要がある。
移動させる作業はロボットの重量が増えていくごとに困難になり、
また運用者が高齢になると作業中の事故などが懸念される。
また、人間が代わりの作業を行うとなると、ロボットを導入したメリットを享受できるのが1フロアだけになってしまう。


そこで、車輪で移動を行い、段差の上り下りを車輪に依存せずに行うロボットを開発すれば、
車輪径や車高は別の問題に最適化しつつ、フロア移動が可能になるのではないかと考えた。
